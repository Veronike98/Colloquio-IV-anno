\documentclass[]{beamer}
% Locale
\usepackage[british]{babel}
\usepackage[english=british]{csquotes} % le paraentesi angolate non sono belle, meglio questo

% LaTeX
\usepackage{geometry}
\usepackage{tikz}
%\usetikzlibrary{quantikz}
\usetikzlibrary{math,fadings} %disegni
\usepackage{tikzpeople} %per fare i personaggini
%\usepackage{callouts} %per i fumetti
\usetikzlibrary{shapes}
\usepackage{tcolorbox} % fare i box colorati
\usepackage{xcolor} % definire colori
\usepackage{tikzsymbols} %per mettere le emoji
\usepackage{adjustbox} %per il disegno in background
\usepackage{transparent}
\usepackage{tikz-cd}


% Math
\usepackage{amsmath}
\usepackage{amssymb}
\usepackage{amsthm}
\usepackage{bbm}
\usepackage{mathrsfs}
\usepackage{centernot} % Serve per fare il not al centro delle frecce
\usepackage{bm} % vettori bold
\usepackage{mathtools} %per le graffe a destra

% Physics
\usepackage{braket}

%per fare le parentesi sulla destra di una lista

%Disegnini
\usepackage{pgfplots}
\usepackage{contour}
\usetikzlibrary{calc,math,decorations.markings,decorations.text,backgrounds,fadings}
\usepgfplotslibrary{fillbetween}
\usetikzlibrary{decorations.pathreplacing,math,calc,tikzmark}

%%% Graffe verticali per il testo
\tikzset{My Node Style/.style={midway, right, xshift=3.0ex, align=left, font=\small, draw=none, thin, text=black}}

\newcommand{\VerticalBrace}[4][]{%
	% #1 = draw options
	% #2 = top mark
	% #2 = bottom mark
	% #4 = label
	\begin{tikzpicture}[overlay,remember picture]
	\tikzmath{coordinate \p,\q;\p=(pic cs:#2);\q=(pic cs:#3);\maxx=max(\px,\qx);}
	\draw[xshift=1ex,decorate,decoration={brace, amplitude=1.5ex}, #1] 
	([yshift=1.5ex]{{pic cs:#2} -| \maxx pt,0})  -- ([yshift=-.5ex]{{pic cs:#3} -| \maxx pt,0})
	node[My Node Style] {#4};
	\end{tikzpicture}
}

% Bibliografia
\usepackage[backend=bibtex,doi=false,isbn=false,url=false]{biblatex}
\addbibresource{bibliografia.bib}

% Colori
\definecolor{bluunipi}{RGB}{25,62,131}
\definecolor{tasselred}{RGB}{130,29,45}
\definecolor{blond}{RGB}{251,231,161}
\definecolor{skincolor}{RGB}{224,177,132}%{255,220,177}
\definecolor{idea}{RGB}{254,231,2} 								%giallo ideee
\definecolor{definition}{RGB}{0,113,145}%{0,0,204}%{51,153,255}%{62,222,217}%{0,204,204}%{29,211,247} %azzurro definizioni
\definecolor{theorem}{RGB}{42,157,143}							 %verde teoremi
\definecolor{titleorange}{RGB}{255,153,0} 					%arancio per i titoletti


% Comandi per i teoremi, definizioni ed esmpi
\theoremstyle{plain}
\newtheorem{thm}{Teorema} % reset theorem numbering for each chapter
\newtheorem{post}{Postulato} % reset theorem numbering for each chapter
\newtheorem{lem}{Lemma}

\theoremstyle{definition}
\newtheorem{defn}{Definizione} % definition numbers are dependent on theorem numbers
\newtheorem{exmp}{Esempio} % same for example numbers
\newtheorem*{ach}{Attenzione!}

% Shortcut
\newcommand{\SigmaX}{\hat{\sigma}^{(x)}}
\newcommand{\SigmaY}{\hat{\sigma}^{(y)}}
\newcommand{\SigmaZ}{\hat{\sigma}^{(z)}}

\newcommand{\Hilb}{\mathcal{H}}
\newcommand{\Id}{\hat{\mathcal{I}}}
\newcommand{\LinSet}{\mathscr{L}}
\newcommand{\ChoiState}[1]{\hat{\rho}_\text{CJ}^{#1}}

\DeclareMathOperator{\Tr}{tr}

%pacchetti per autisticare
\usepackage{duckuments}
\usepackage{halloweenmath}


%box carini
\newenvironment<>{ideablock}[1]{%
	\setbeamercolor{block title}{fg=white,bg=idea}%
	\begin{block}#2{#1}}{\end{block}}

\newenvironment<>{defblock}[1]{%
	\setbeamercolor{block title}{fg=white,bg=definition}%
	\begin{block}#2{#1}}{\end{block}}

\newenvironment<>{theoblock}[1]{%
	\setbeamercolor{block title}{fg=white,bg=theorem}%
	\begin{block}#2{#1}}{\end{block}}


% due nuovi comandi per allineare le cose bene
\newcommand\parallelcontent[2]{
	\begin{columns}[t]
		\column{0.5\textwidth} #1
		\column{0.5\textwidth} #2
	\end{columns}
}
\newcommand\parallelitem[2]{
	\parallelcontent
	{\begin{itemize} \item #1 \end{itemize}}
	{\begin{itemize} \item #2 \end{itemize}}
}


\usepackage[utf8]{inputenc}

\usetheme{Frankfurt}
\setbeamercovered{transparent} %per rendere sbiadito cio' che verra'
\setbeamertemplate{navigation symbols}{}

\usecolortheme{wolverine} %wolverine, seahorse, crane, beaver


%Information to be included in the title page:
\title[Penrose's Singularity Theorem]
{Penrose's Singularity Theorem}
%\subtitle[it]{\texttt{Inserisci sottotitolo}}
\author[Veronica Sacchi]{Veronica Sacchi \\[1ex]
	{\small Advisor: Enrico Trincherini}}      
\institute[SNS]{\includegraphics[height=1.45cm]{Immagini/logoSNS/orizzontale-colore/orizzontale-colore.jpg}}
\date[30-04-2021]{30th April 2021}
%\logo{\includegraphics[height=1.1cm]{Immagini/logoSNS/tondo-colore/tondo-colore.png}}

\begin{document}	
	
	\usebackgroundtemplate{
		\adjustbox{height=0.8\paperheight, raise=-9cm, right=15cm}{
			\transparent{0.05}
			\includegraphics{Immagini/logoSNS/solo_logo.jpeg}
		}
	}
	
	\begin{frame}
	\titlepage
	\end{frame}
	
	\section{Basic definitions}
	\subsection{Globally Hyperbolic Spaces}
	
%%%%%%%%%%%%%%%%%%%%%%%%%%%%%%%%%%%%%%%%%%%%% Slide 2
	\begin{frame}
	\frametitle{First definitions}
	\begin{defblock}{Spacelike hypesurface}
		A hypersurface \(S\) is \emph{spacelike} if the induced metric has Euclidean signature. 
	\end{defblock}
	\begin{defblock}{Achronal hypesurface}
		\(S\) is said to be \emph{achronal} if there is no \emph{timelike} path in \(M\) connecting \(2\) distinct points \(q,p \in S\).
	\end{defblock}
	\pause
	
	\begin{columns}[T]
	\column{0.43\textwidth}
	\begin{figure}
		\begin{tikzpicture}
			\coordinate (a) at (0,0);
			\coordinate (b) at (1.4,0);
			\coordinate (c) at (0,2.2);
			\coordinate (d) at (1.4,2.2);
			\coordinate (e) at (1.4, 0.8);
			\coordinate (f) at (0, 1.6);
			\coordinate (g) at (0.7, 1.9);
			\coordinate (h) at (0.7, 0.20);
			\draw (a)--(c);
			\draw (b)--(d);
			\draw[dashed] (0,0) arc (180:0:0.7 and 0.3);
			\draw (0,0) arc (180:360:0.7 and 0.3);
			%\draw[dashed] (0.7, 0) ellipse (0.7 and 0.3);
			\draw (0.7, 2.2) ellipse (0.7 and 0.3);
			\draw[titleorange] (a) to [out = 15, in = -115] (e);
			\draw[dashed, titleorange] (e) to [out = 125, in = -20] (f);
			\draw[titleorange] (f) to [out = 15, in = -145] (g);
			\draw[thick, definition] (h) -- (g);
			\foreach \i in {h,g} {\fill[definition] (\i) circle(.8pt);}
			\node[below] at (h) {\footnotesize\(p\)};
			\node[above] at (g) {\footnotesize\(q\)};
			\node[left] at (f) {\footnotesize\(S\)};
			
		\end{tikzpicture}
		%\includegraphics[scale=0.5]{example-image-duck}
%		\caption{\(t = \epsilon\phi\) is not achronal but can be spacelike.}
	\end{figure}
	\column{0.57\textwidth}
	%todo: spostare piu' in su il blocco e decidere che tipo di blocco usare davvero
	\begin{ideablock}{Attention}
		\centering
		Spacelike \(\nRightarrow\) Achronal
	\end{ideablock}
 	\vskip -10pt
	\begin{figure}
		\caption{\(t = \epsilon\phi\) is not achronal but can be spacelike.}
	\end{figure}
	\end{columns}

	\end{frame}
	
%%%%%%%%%%%%%%%%%%%%%%%%%%%%%%%%%%%%%%%%%%%%% Slide 3
	\begin{frame}[fragile]
		\frametitle{Globally Hyperbolic Spaces}
		\begin{theoblock}{}
			
			\begin{columns}
				\column{0.4\textwidth}
				\centering
				\(p,q\in S\) spacelike, achronal \& codim\(S= 1\)
				\column{0.6\textwidth}
				\begin{columns}
					\column{0.05\textwidth} \centering \vfill \(\implies\)
					\column{0.55\textwidth}
					\centering
					there is no \emph{causal} path from \(p\) to \(q\).
				\end{columns}
			\end{columns}
		\end{theoblock}
	
	\begin{defblock}{Cauchy Hypersurface}
		\(S\) is a \emph{Cauchy} hypersurface if it is \emph{achronal}, \emph{spacelike} \& if 
		\(\forall p \notin S\) any inextendible \emph{causal} path through \(p\) intersects \(S\).
	\end{defblock}
		\pause
		
	\begin{columns}
		\column{0.6\textwidth}
		\centering
		%\includegraphics[scale=0.7]{example-image-duck}
		\begin{tikzpicture}[x={(1.5,0)},y={(0,1.5)},z={(.5,.5)},
		pics/geodesic/.style n args={3}{code={
				% #1 = draw options
				% #2 = first path
				% #3 = second path
				\draw[path fading=north,#1,postaction={decorate,decoration={markings,mark=at position .75 with {\arrowreversed{stealth}}}}] #2 coordinate (continue) pic[pics/code={\fill circle(0.7pt);}]{};
				\path[name path=geodesic] #3;
				\begin{scope}[on background layer={}]
				\draw[#1,dash pattern=on .9pt off .9pt,intersection segments={of=geodesic and lower boundary,sequence={L1}}];
				\draw[path fading=south,#1,intersection segments={of=geodesic and lower boundary,sequence={L2}}];
				\end{scope}
		}}]
		\colorlet{surfacecol}{orange}
		\coordinate (a) (0,0,0);
		\coordinate (b) at (2,-.3,0);
		\coordinate (c) at (2,0,2);
		\coordinate (d) at (0,0,2);
		\tikzmath{
			\outb0=140;\inb0=5;\oute0=210;\ine0=30;
			\outb1=50;\inb1=-100;\oute1=80;\ine1=-120;
		}
		\foreach \p/\q/\i/\n in {b/a/0/b,c/d/0/e,b/c/1/b,a/d/1/e} {
			\path[decorate,decoration={markings,mark=between positions 0 and 1 step 0.0999 with {\tikzmath{\j=int(\pgfkeysvalueof{/pgf/decoration/mark info/sequence number}-1);}\coordinate (p-\i-\j-\n);}}]
			(\p) to[out=\csname out\n\endcsname\i,in=\csname in\n\endcsname\i] (\q);
		}
		\fill[surfacecol!20,opacity=.5] (b) to[out=\outb0,in=\inb0] (a) to[out=\oute1,in=\ine1] (d) to[out=\ine0,in=\oute0] (c) to[out=\inb1,in=\outb1] (b);
		\foreach \pb/\qb/\pe/\qe/\i/\j in {b/c/a/d/1/0,b/a/c/d/0/1} {
			\foreach \tt[evaluate=\tt as \t using 0.0999*\tt] in {1,...,10} {
				\tikzmath{
					\myout={\outb{\i}*(1-\t)+\oute{\i}*\t};
					\myin={\inb{\i}*(1-\t)+\ine{\i}*\t};
					\myop=0.5;
					\mylw=0.35;
					if \tt==10 then {
						\myop=1;\mylw=0.6;
					};
				}
				\draw[surfacecol,opacity=\myop,line width=\mylw] (p-\j-\tt-b) to[out=\myout,in=\myin] (p-\j-\tt-e);
		}}
		\draw[surfacecol,line width=1.2,name path=lower boundary] (a) to[out=\inb0,in=\outb0] (b) to[out=\outb1,in=\inb1] (c);
		\tikzset{
			geodesic style/.style={green!80!black,line width=.6}
		}
		\pic{geodesic={geodesic style}{(.2,1.3,.2) to[out=-70,in=110] (.2,.1,.2)}{(continue) to[out=-70,in=90] (.3,-.8,.3)}};
		\pic{geodesic={geodesic style}{(.9,1.3,1) to[out=-80,in=100] (.8,.1,1)}{(continue) to[out=-80,in=80] (.8,-.8,1)}};
		\pic{geodesic={geodesic style}{(1.8,1.3,.3) to[out=-80,in=60] (1.8,-.1,.3)}{(continue) to[out=-120,in=80] (1.6,-.8,.3)}};
		\pic{geodesic={geodesic style}{(1.8,1.3,1.5) to[out=-90,in=120] (1.7,-.2,1.8)}{(continue) to[out=-60,in=90] (1.9,-.8,1.9)}};
		\pic{geodesic={geodesic style}{(1.4,1.5,1) to[out=-90,in=108] (1.17,.1,1.5)}{(continue) to[out=-72,in=110] (1.95,-.8,.74)}};
		\pic{geodesic={geodesic style}{(.7,1.5,.5) to[out=-45,in=110] (.5,.3,.9)}{(continue) to[out=-70,in=90] (.8,-.7,.3)}};
		\pic{geodesic={geodesic style}{(1.5,1.7,.2) to[out=-70,in=110] (.8,.1,.2)}{(continue) to[out=-70,in=90] (.5,-.8,.3)}};
		
		\node[left] at (a) {\footnotesize\(S\)};
		\end{tikzpicture}
		

	\column{0.4\textwidth}
	\begin{defblock}{Globally Hyperbolic Space}
		\(M\) is said to be \emph{Globally Hyperbolic} if it contains a Cauchy hypersurface \(S\).
	\end{defblock}
	\end{columns}
		
	\end{frame}
	
%%%%%%%%%%%%%%%%%%%%%%%%%%%%%%%%%%%%%%%%%%%%% Slide 4

	\begin{frame}
		\begin{columns}
			\column{0.6\textwidth}
			\begin{defblock}{Weak causality}
				There is no nonconstant causal curve from a point \(q\in M\) to iself
			\end{defblock}
		\pause
		\centering{\LARGE\(\Uparrow\)}
		\begin{defblock}{Strong causality}
			Every point \(q\in M\) has an arbitrarily small neighborhood V such that any causal curve between points \(p,p'\in V\) is entirely contained in \(V\). 
		\end{defblock}
	
		{\huge\(\Uparrow\)}
		\begin{theoblock}{}
			\centering
			Global Hyperbolicity
		\end{theoblock}
		\column{0.4\textwidth}
		\begin{figure}
			%\includegraphics[scale=0.7]{example-image-duck}
			\begin{tikzpicture}[scale=0.9]
				\coordinate (a) at (0,0);
				\coordinate (b) at (2,2);
				\coordinate (c) at (0,4);
				\coordinate (d) at (-2, 2);
				\coordinate (q) at (0, 1.5);
				\draw[thick, definition] (a) to [out = 70, in = -160, looseness = 1] (b);
				\draw[thick, definition] (b) to [out = 160, in = -70, looseness = 1] (c);
%				\draw[thick, definition] (d) to [out = 20, in = -110, looseness = 0.5] (c);
%				\draw[thick, definition] (d) to [out = -20, in = 110, looseness = 0.5] (a);
				
				\colorlet{surfacecol}{definition}
				
				
				\tikzmath{
					\outb0=-160;\inb0=70;\oute0=-110;\ine0=20;
					\outb1=160;\inb1=-70;\oute1=130;\ine1=-20;
				}
				\foreach \p/\q/\i/\n in {b/a/0/b,c/d/0/e,b/c/1/b,a/d/1/e} {
					\path[decorate,decoration={markings,mark=between positions 0 and 1 step 0.0999 with {\tikzmath{\j=int(\pgfkeysvalueof{/pgf/decoration/mark info/sequence number}-1);}\coordinate (p-\i-\j-\n);}}]
					(\p) to[out=\csname out\n\endcsname\i,in=\csname in\n\endcsname\i] (\q);
				}
				\fill[surfacecol!20,opacity=.5] (b) to[out=\outb0,in=\inb0] (a) to[out=\oute1,in=\ine1] (d) to[out=\ine0,in=\oute0] (c) to[out=\inb1,in=\outb1] (b);
				\foreach \pb/\qb/\pe/\qe/\i/\j in {b/c/a/d/1/0,b/a/c/d/0/1} {
					\foreach \tt[evaluate=\tt as \t using 0.0999*\tt] in {1,...,10} {
						\tikzmath{
							\myout={\outb{\i}*(1-\t)+\oute{\i}*\t};
							\myin={\inb{\i}*(1-\t)+\ine{\i}*\t};
							\myop=0.5;
							\mylw=0.35;
							if \tt== 10 then {
								\myop=1;\mylw=0.6;
							};
						}
						\draw[surfacecol,opacity=\myop,line width=\mylw] (p-\j-\tt-b) to[out=\myout,in=\myin] (p-\j-\tt-e);
				}}
				
				\draw[thick, titleorange, postaction={decorate,decoration={markings,mark=at position .25 with {\arrow{stealth}}}}] (0, 3) circle (1.5);
				\fill[definition] (q) circle(1.1pt);
				\node[below] at (q) {\(q\)};
				\node[below] at (a) {\(V\)};
				
			\end{tikzpicture}
			\caption{Strong causality implies weak causality.}
		\end{figure}
		%todo: disegna l'intorno V di q con una curva rossa chiusa che esce, per far capire che strong => weak
		\end{columns}
	\end{frame}	
	
	
%%%%%%%%%%%%%%%%%%%%%%%%%%%%%%%%%%%%%%%%%%%%%%% Slide 5
	\section{Hawking's singularity theorem}
	
	\subsection{The Riemannian Case}
	\begin{frame}
	
	\frametitle[Slide1]{The Riemannian Case}

	\begin{ideablock}{Key-question}
		Is a geodesic always the shortest distance between two points?
	\end{ideablock}

	\pause
	
	\begin{columns}
		\column{0.4\textwidth}
		\begin{defblock}{Focal Point}
			If a geodesic segment \(qq'\) can be displaced - at least in first order - to a nearby geodesic from \(q\) to \(q'\), then \(q'\) is said to be a \emph{focal} point. 
		\end{defblock}
		\column{0.6\textwidth}
		\begin{figure}
			\begin{tikzpicture}[scale = 1.5]
			\coordinate(q1) at(0,0);
			\coordinate(q) at(-2,1);
			\coordinate(p) at(2,0);
			\foreach \i in {p,q,q1} {\fill[black] (\i) circle(.8pt);}
			\draw[green] (p) -- (q);
			\draw[titleorange] (q1) -- (q);
			\draw[definition] (p) -- (q1) to [out = 180, in = -45] (q);
			\node[below] at(q1) {\(q'\)};
			\node[left] at(q)  {\(q\)};
			\node[below] at(p) {\(p\)};
			
			\end{tikzpicture}
			%	\includegraphics[scale=1.0]{example-image-duck}
			\caption{a geodesic emanating from \(q\) is no longer length minimizing once it is continued past its focal point.}
		\end{figure}
		
	\end{columns}
	
		\end{frame}
%todo: ricordarsi di dire le cose in questo elenco puntato	
%	%Slide 2
%	\begin{frame}
%	
%
%	\pause
%	
%	\begin{theoblock}{}
%		We should conclude that a lenght minimizing geodesic mustn't have any focal point.
%	\end{theoblock}
%	
%	\pause
%	
%	\begin{itemize}
%		\item<2-> This is a necessary but not sufficient condition.
%		\item<3-> The definition can be extended to geodesics connecting point \(q\) to a submanifold \(W\). 
%		
%		Now a point \(q'\) is said to be \emph{focal} if there exist points \(p,p'\in W\) such that:
%		\begin{enumerate}
%			\item \(qq'p\) and \(qq'p'\) are both geodesics.
%			\item \(qq'p\) and \(qq'p'\) are both orthogonal to W.
%		\end{enumerate}
%	\end{itemize}
%	
%
%	\end{frame}


%%%%%%%%%%%%%%%%%%%%%%%%%%%%%%%%%%%%%%%%%%%%% Slide 5
	\subsection{The Lorentz SIgnature Analog}
	\begin{frame}
		\frametitle[Slide3]{The Lorentz Signature Analog}
		\begin{ideablock}{Key-Idea}
			In timelike objects distance is replaced by the elapsed proper time.
		\end{ideablock}
	\pause
	
	\vskip 7pt
	\parallelcontent{\textcolor{titleorange}{\textbf{Riemannian Signature}}}{\textcolor{titleorange}{\textbf{Lorentz Signature}}}
	\parallelitem{locally minimizing distance.}{locally maximizing proper time s.}
	\parallelitem{\(q'\) is focal if there are \(2\) nearby geodesic segments \(qq'\)}{\(q'\) is focal if there are \(2\) nearby \emph{timelike} geodesic segments \(qq'\)}
	\end{frame}
%	%Slide4
%	\begin{frame}
%	
%	\begin{center}
%		A curve \(\gamma\) is extremizing if
%	\end{center}
%	\pause
%	\vskip 7pt
%	\parallelcontent{\textcolor{titleorange}{\textbf{Riemannian Signature}}}{\textcolor{titleorange}{\textbf{Lorentz Signature}}}
%	\parallelitem{\(\gamma\) is a geodesic;}{\(\gamma\) is a \emph{timelike} geodesic;}
%	\parallelitem{\(\gamma\) is orthogonal;}{\(\gamma\) is orthogonal;}
%	\parallelitem{\(\gamma\) doesn't contain any focal point.}{\(\gamma\) doesn't contain any focal point.}
%	\end{frame}
%
%	\subsection{Raychaudhuri's equation}
%	%Slide4

%%%%%%%%%%%%%%%%%%%%%%%%%%%%%%%%%%%%%%%%%%%%%%%%%%%% Slide 6
	\begin{frame}[fragile]
		\frametitle{Coordinate extension}
		\begin{columns}
			\column{0.4\textwidth}
			\centering
			\begin{tikzpicture}[x={(1.5,0)},y={(0,1.5)},z={(.5,.5)},
			pics/geodesic/.style n args={3}{code={
					% #1 = draw options
					% #2 = first path
					% #3 = second path
					\draw[path fading=north,#1,postaction={decorate,decoration={markings,mark=at position .75 with {\arrowreversed{stealth}}}}] #2 coordinate (continue) pic[pics/code={\fill circle(0.7pt);}]{};
					\path[name path=geodesic] #3;
					\begin{scope}[on background layer={}]
					\draw[#1,dash pattern=on .9pt off .9pt,intersection segments={of=geodesic and lower boundary,sequence={L1}}];
					\draw[path fading=south,#1,intersection segments={of=geodesic and lower boundary,sequence={L2}}];
					\end{scope}
			}}]
			\colorlet{surfacecol}{orange}
			\coordinate (a) (0,0,0);
			\coordinate (b) at (2,-.3,0);
			\coordinate (c) at (2,0,2);
			\coordinate (d) at (0,0,2);
			\tikzmath{
				\outb0=140;\inb0=5;\oute0=210;\ine0=30;
				\outb1=50;\inb1=-100;\oute1=80;\ine1=-120;
			}
			\foreach \p/\q/\i/\n in {b/a/0/b,c/d/0/e,b/c/1/b,a/d/1/e} {
				\path[decorate,decoration={markings,mark=between positions 0 and 1 step 0.0999 with {\tikzmath{\j=int(\pgfkeysvalueof{/pgf/decoration/mark info/sequence number}-1);}\coordinate (p-\i-\j-\n);}}]
				(\p) to[out=\csname out\n\endcsname\i,in=\csname in\n\endcsname\i] (\q);
			}
			\fill[surfacecol!20,opacity=.5] (b) to[out=\outb0,in=\inb0] (a) to[out=\oute1,in=\ine1] (d) to[out=\ine0,in=\oute0] (c) to[out=\inb1,in=\outb1] (b);
			\foreach \pb/\qb/\pe/\qe/\i/\j in {b/c/a/d/1/0,b/a/c/d/0/1} {
				\foreach \tt[evaluate=\tt as \t using 0.0999*\tt] in {1,...,10} {
					\tikzmath{
						\myout={\outb{\i}*(1-\t)+\oute{\i}*\t};
						\myin={\inb{\i}*(1-\t)+\ine{\i}*\t};
						\myop=0.5;
						\mylw=0.35;
						if \tt==10 then {
							\myop=1;\mylw=0.6;
						};
					}
					\draw[surfacecol,opacity=\myop,line width=\mylw] (p-\j-\tt-b) to[out=\myout,in=\myin] (p-\j-\tt-e);
			}}
			\draw[surfacecol,line width=1.2,name path=lower boundary] (a) to[out=\inb0,in=\outb0] (b) to[out=\outb1,in=\inb1] (c);
			\tikzset{
				geodesic style/.style={green!80!black,line width=.6}
			}
			\pic{geodesic={geodesic style}{(.2,1.3,.2) to[out=-70,in=110] (.2,.1,.2)}{(continue) to[out=-70,in=90] (.3,-.8,.3)}};
			\pic{geodesic={geodesic style}{(.9,1.3,1) to[out=-80,in=100] (.8,.1,1)}{(continue) to[out=-80,in=80] (.8,-.8,1)}};
			\pic{geodesic={geodesic style}{(1.8,1.3,.3) to[out=-80,in=60] (1.8,-.1,.3)}{(continue) to[out=-120,in=80] (1.6,-.8,.3)}};
			\pic{geodesic={geodesic style}{(1.8,1.3,1.5) to[out=-90,in=120] (1.7,-.2,1.8)}{(continue) to[out=-60,in=90] (1.9,-.8,1.9)}};
			\pic{geodesic={geodesic style}{(1.4,1.5,1) to[out=-90,in=108] (1.17,.1,1.5)}{(continue) to[out=-72,in=110] (1.95,-.8,.74)}};
			\pic{geodesic={geodesic style}{(.7,1.5,.5) to[out=-45,in=110] (.5,.3,.9)}{(continue) to[out=-70,in=90] (.8,-.7,.3)}};
			\pic{geodesic={geodesic style}{(1.5,1.7,.2) to[out=-70,in=110] (.8,.1,.2)}{(continue) to[out=-70,in=90] (.5,-.8,.3)}};
			
			\node[left] at (a) {\footnotesize\(S\)};
			\end{tikzpicture}
			%\includegraphics[width=\linewidth]{example-image-duck}
			\column{0.6\textwidth}
			Given a spacelike hypersurface \(S\) we can extend its local set of coordinates:
			\vfill
			\begin{center}
				\begin{tikzcd}[column sep = tiny]
					\dim S=d\rar&\vec{x}=(x_1,\ldots,x_d)\dar[dashed]\\
					\dim M=d+1\rar\uar&(t,\vec{x})
				\end{tikzcd}
			\end{center}
%			\begin{align*}
%				\dim S =d \longrightarrow &\vec{x} = (x_1, \ldots, x_d)\\
%				 \uparrow \quad \quad& \quad\downarrow \\
%				 \dim M = d+1 \longrightarrow & (t, \vec{x}).
%			\end{align*}
		\end{columns}
	\vskip 11pt
	\pause
	\begin{theoblock}{Consequence}
		\centering \(ds^2 = -dt^2 + g_{ij}(t,\vec{x})dx^idx^j\).
	\end{theoblock}
	
	\end{frame}

%%%%%%%%%%%%%%%%%%%%%%%%%%%%%%%%%%%%%%%%%%%%% Slide 7
	\begin{frame}
		\frametitle{Raychaudhuri's Equation}
		\begin{ideablock}{Einstein equation}
		\centering	\(R_{tt} = 8\pi G \underbrace{\left(T_{tt} - \frac{1}{D - 2}g_{tt}T_{\alpha}^{\alpha}\right) }_{\hat{T}_{tt}}\).
		\end{ideablock}

	\pause
	\[
	\begin{rcases}
	V := \sqrt{\det g}\\
	\theta := \frac{\dot{V}}{V} = \frac{1}{2} \Tr[g^{-1}\dot{g}]\\
	\sigma^i_j := \frac{1}{2}\left(g^{ik}\dot{g}_{kj} - \frac{1}{d}\delta^i_j\Tr[g^{-1}\dot{g}]\right)
	\end{rcases}
	\implies
	\]
	
	\begin{theoblock}{Raychaudhuri's equation}
		 \centering \(\partial_t\left(\frac{\dot{V}}{V}\right) + \frac{1}{d}\left(\frac{\dot{V}}{V}\right)^2 =-\Tr[\sigma^2] - 8\pi G\hat{T}_{tt}.\)
	\end{theoblock}

	\end{frame}

%%%%%%%%%%%%%%%%%%%%%%%%%%%%%%%%%%%%%%%%%%%%% Slide 8
	\begin{frame}[fragile]
	\frametitle{Two Consequences}
	\begin{defblock}{Strong energy condition}
		\centering	\(\hat{T}_{tt}\ge 0 \implies \partial_t\left(\frac{1}{\theta}\right)\ge \frac{1}{d}\)
	\end{defblock}

	\begin{ideablock}{Attention}
		The strong energy condition is \emph{not} satisfied by a positive cosmological constant! 
	\end{ideablock}
	
	\vskip 17pt
	\pause
	\begin{columns}
		\column{0.7\textwidth}
		If \(\exists p \in S\) such that \(\theta = - w < 0\) then
		\begin{defblock}{}
			\centering	\(\frac{\dot{V}}{V} \le -\left(\frac{1}{w} - \frac{t}{d}\right)^{-1}\implies V(t)\rightarrow 0  \) if \(t \rightarrow \frac{d}{w}\)
		\end{defblock}
		\column{0.3\textwidth}
		\begin{tikzpicture}[x={(1.0,0)},y={(0,1.0)},z={(.5,.5)},
		pics/geodesic/.style n args={3}{code={
				% #1 = draw options
				% #2 = first path
				% #3 = second path
				\draw[path fading=north,#1,postaction={decorate,decoration={markings,mark=at position .75 with {\arrowreversed{stealth}}}}] #2 coordinate (continue) pic[pics/code={\fill circle(0.7pt);}]{};
				\path[name path=geodesic] #3;
				\begin{scope}[on background layer={}]
				\draw[#1,dash pattern=on .9pt off .9pt,intersection segments={of=geodesic and lower boundary,sequence={L1}}];
				\draw[path fading=south,#1,intersection segments={of=geodesic and lower boundary,sequence={L2}}];
				\end{scope}
		}}]
		\colorlet{surfacecol}{orange}
		\coordinate (a) (0,0,0);
		\coordinate (b) at (2,-.3,0);
		\coordinate (c) at (2,0,2);
		\coordinate (d) at (0,0,2);
		\tikzmath{
			\outb0=140;\inb0=5;\oute0=210;\ine0=30;
			\outb1=50;\inb1=-100;\oute1=80;\ine1=-120;
		}
		\foreach \p/\q/\i/\n in {b/a/0/b,c/d/0/e,b/c/1/b,a/d/1/e} {
			\path[decorate,decoration={markings,mark=between positions 0 and 1 step 0.0999 with {\tikzmath{\j=int(\pgfkeysvalueof{/pgf/decoration/mark info/sequence number}-1);}\coordinate (p-\i-\j-\n);}}]
			(\p) to[out=\csname out\n\endcsname\i,in=\csname in\n\endcsname\i] (\q);
		}
		\fill[surfacecol!20,opacity=.5] (b) to[out=\outb0,in=\inb0] (a) to[out=\oute1,in=\ine1] (d) to[out=\ine0,in=\oute0] (c) to[out=\inb1,in=\outb1] (b);
		\foreach \pb/\qb/\pe/\qe/\i/\j in {b/c/a/d/1/0,b/a/c/d/0/1} {
			\foreach \tt[evaluate=\tt as \t using 0.0999*\tt] in {1,...,10} {
				\tikzmath{
					\myout={\outb{\i}*(1-\t)+\oute{\i}*\t};
					\myin={\inb{\i}*(1-\t)+\ine{\i}*\t};
					\myop=0.5;
					\mylw=0.35;
					if \tt==10 then {
						\myop=1;\mylw=0.6;
					};
				}
				\draw[surfacecol,opacity=\myop,line width=\mylw] (p-\j-\tt-b) to[out=\myout,in=\myin] (p-\j-\tt-e);
		}}
		\draw[surfacecol,line width=1.2,name path=lower boundary] (a) to[out=\inb0,in=\outb0] (b) to[out=\outb1,in=\inb1] (c);
		\tikzset{
			geodesic style/.style={green!80!black,line width=.6}
		}
%		\pic{geodesic={geodesic style}{(.2,1.3,.2) to[out=-70,in=110] (.2,.1,.2)}{(continue) to[out=-70,in=90] (.3,-.8,.3)}};
%		\pic{geodesic={geodesic style}{(.9,1.3,1) to[out=-80,in=100] (.8,.1,1)}{(continue) to[out=-80,in=80] (.8,-.8,1)}};
		\pic{geodesic={geodesic style}{(1.4,1.4,.4) to[out=-80,in=60] (1.8,-.1,.3)}{(continue) to[out=-120,in=80] (1.6,-.8,.3)}};
%		\pic{geodesic={geodesic style}{(1.8,1.3,1.5) to[out=-90,in=120] (1.7,-.2,1.8)}{(continue) to[out=-60,in=90] (1.9,-.8,1.9)}};
%		\pic{geodesic={geodesic style}{(1.4,1.5,1) to[out=-90,in=108] (1.17,.1,1.5)}{(continue) to[out=-72,in=110] (1.95,-.8,.74)}};
		\pic{geodesic={geodesic style}{(1.4,1.4,.4) to[out=-45,in=110] (.7,.1,.9)}{(continue) to[out=-70,in=90] (.8,-.7,.3)}};
%		\pic{geodesic={geodesic style}{(1.5,1.7,.2) to[out=-70,in=110] (.8,.1,.2)}{(continue) to[out=-70,in=90] (.5,-.8,.3)}};

		\node[left] at (d) {\footnotesize\(S\)};
		\end{tikzpicture}
	\end{columns}
	 
	\end{frame}

%%%%%%%%%%%%%%%%%%%%%%%%%%%%%%%%%%%%%%%%%%%%% Slide 9
	\subsection{Hawking's theorem}

	\begin{frame}
		\frametitle{Hawking's theorem}
		\begin{itemize}
			\item M is smooth and time-oriented;
			\item M is Globally Hyperbolic;	\tikzmark{top 1}
			\item S is a Cauchy Hypersurface;\tikzmark{bottom 1}\tikzmark{top 2}
			\item \(\forall p \in S \quad \quad \frac{\dot{V}}{V}\ge dh_{\text{min}} \).\tikzmark{bottom 2}
		\end{itemize}
	
	\VerticalBrace[ultra thick, titleorange]{top 1}{bottom 1}{\(\gamma\) geodesic maximizing \(s\implies\)\\\(\perp\) S and without focal points.}
	%{\textcolor{titleorange}{\(\gamma\) geodesics maximizing \(s\implies\)\\\(\perp\) S and without focal points.}}
	\VerticalBrace[ultra thick, idea]{top 2}{bottom 2}{\newline \\ \newline \\Every \(\gamma\) has a focal point.}
	%{\textcolor{idea}{Every \(\gamma\) has a focal point.}}

	
	\begin{theoblock}{Hawking's Singularity Theorem}
		\(\nexists p \in M\) to the past of S by a proper time greater than \(\frac{1}{h_{\text{min}}}\).
		
		Equivalently, \(M\) is \emph{timelike} geodesically incomplete.
	\end{theoblock}
		

	\end{frame}

%%%%%%%%%%%%%%%%%%%%%%%%%%%%%%%%%%%%%%%%%%%%%%%%%%%%%%Slide 10
\section{The null analog}
\begin{frame}
	\frametitle{The null analog}
	\begin{columns}
		\column{0.7\textwidth}
		\begin{defblock}{Promptness}
			A causal path \(\ell\) from \(q\) to \(p\) is \emph{prompt} if there is no causal path \(\ell'\) from \(q\) to a point \(r\) just to the past of \(p\).
		\end{defblock}
		\column{0.3\textwidth}
		\begin{tikzpicture}
			\coordinate (q) at (0, 0);
			\coordinate (p) at (2, 2);
			\coordinate (r) at (2, 1); 
			\draw[theorem, postaction={decorate,decoration={markings,mark=at position .5 with {\arrow{stealth}}}}] (q) -- (p);
			\draw[titleorange, dashed, postaction={decorate,decoration={markings,mark=at position .5 with {\arrow{stealth}}}}] (q) -- (r);
			\draw[titleorange, postaction={decorate,decoration={markings,mark=at position .5 with {\arrow{stealth}}}}] (r) -- (p);
			\foreach \i in {p,q,r} {\fill[definition] (\i) circle(.8pt);}
			\node[below] at (q) {\(q\)};
			\node[below] at (r) {\(r\)};
			\node[above] at (p) {\(p\)};
		\end{tikzpicture}
	\end{columns}
	\pause
	
	\begin{enumerate}
		\item \(\ell\) prompt \(\implies\) \(\ell\) is an orthogonal null geodesic, with no focal points.
		\item Strong causality \(\implies\) any null geodesics has a prompt initial segment.
	\end{enumerate}
	\end{frame}


%%%%%%%%%%%%%%%%%%%%%%%%%%%%%%%%%%%%%%%%%%%%%%%%%%%%%%Slide 11
	\begin{frame}
	\begin{theoblock}{Null Raychaudhuri equation}
		Extending coordinates from a spacelike submanifold \(W\) of codimension \(2\) we get:
		\[\dot{\theta} + \frac{\theta^2}{D-2} = -\Tr[\sigma^2] - 8\pi GT_{uu}\]
	\end{theoblock}
	\pause
	\begin{defblock}{Null energy condition}
		For any null vector \(v^{\mu}\)
		\[v^{\mu}v^{\nu}T_{\mu\nu} \ge 0\]
	\end{defblock}
	\pause
	\begin{defblock}{Trapped Surfaces}
		A \emph{trapped surface} is a codimension \(2\), spacelike, submanifold \(W\) such that the expansion of each family of orthogonal future-going null geodesics is everywhere negative.
	\end{defblock}
	\end{frame}


%%%%%%%%%%%%%%%%%%%%%%%%%%%%%%%%%%%%%%%%%%%%%%%%%%%%%%Slide 13
\subsection{Penrose's Theorem}
\begin{frame}
	\frametitle{Penrose's SingularityTheorem}
	\begin{theoblock}{Penrose's theorem}
		\begin{columns}
			\column{0.45\textwidth}
			\begin{itemize}
				\item M is globally hyperbolic
				\item M has a non-compact Cauchy hypersurface
				\item The Classical Einstein equations hold
				\item Null Energy condition holds
				\item Exists a compact trapped surface
			\end{itemize}
		\column{0.55\textwidth}
			\begin{columns}
				\column{0.05\textwidth}
				{\Large\(\implies\)}
				\column{0.5\textwidth}
				M is not \emph{null geodesically} complete.
		\end{columns}
		\end{columns}
	\end{theoblock}
\end{frame}

%%%%%%%%%%%%%%%%%%%%%%%%%%%%%%%%%%%%%%%%%%%%%%%%%%%%%%Slide 10
\begin{frame}[fragile]
	\begin{columns}
		\column{0.5\textwidth}
		\centering
		%\includegraphics[scale=0.7]{example-image-duck}
		%todo: mettici disegno tipo fig.32 pag. 49, magari con sotto una superficie S nello stile di fig.8 a pag.17
		\begin{tikzpicture}[x={(1,0)},y={(0,1)},z={(.5,.5)},
		pics/geodesic/.style n args={3}{code={
				% #1 = draw options
				% #2 = first path
				% #3 = second path
				\draw[path fading=north,#1,postaction={decorate,decoration={markings,mark=at position .75 with {\arrowreversed{stealth}}}}] #2 coordinate (continue) pic[pics/code={\fill circle(0.7pt);}]{};
				\path[name path=geodesic] #3;
				\begin{scope}[on background layer={}]
				\draw[#1,dash pattern=on .9pt off .9pt,intersection segments={of=geodesic and lower boundary,sequence={L1}}];
				\draw[path fading=south,#1,intersection segments={of=geodesic and lower boundary,sequence={L2}}];
				\end{scope}
		}}]
		\colorlet{surfacecol}{orange}
		\coordinate (a) (0,0,0);
		\coordinate (b) at (2,-.3,0);
		\coordinate (c) at (2,0,2);
		\coordinate (d) at (0,0,2);
		\tikzmath{
			\outb0=140;\inb0=5;\oute0=210;\ine0=30;
			\outb1=50;\inb1=-100;\oute1=80;\ine1=-120;
		}
		\foreach \p/\q/\i/\n in {b/a/0/b,c/d/0/e,b/c/1/b,a/d/1/e} {
			\path[decorate,decoration={markings,mark=between positions 0 and 1 step 0.0999 with {\tikzmath{\j=int(\pgfkeysvalueof{/pgf/decoration/mark info/sequence number}-1);}\coordinate (p-\i-\j-\n);}}]
			(\p) to[out=\csname out\n\endcsname\i,in=\csname in\n\endcsname\i] (\q);
		}
		\fill[surfacecol!20,opacity=.5] (b) to[out=\outb0,in=\inb0] (a) to[out=\oute1,in=\ine1] (d) to[out=\ine0,in=\oute0] (c) to[out=\inb1,in=\outb1] (b);
		\foreach \pb/\qb/\pe/\qe/\i/\j in {b/c/a/d/1/0,b/a/c/d/0/1} {
			\foreach \tt[evaluate=\tt as \t using 0.0999*\tt] in {1,...,10} {
				\tikzmath{
					\myout={\outb{\i}*(1-\t)+\oute{\i}*\t};
					\myin={\inb{\i}*(1-\t)+\ine{\i}*\t};
					\myop=0.5;
					\mylw=0.35;
					if \tt==10 then {
						\myop=1;\mylw=0.6;
					};
				}
				\draw[surfacecol,opacity=\myop,line width=\mylw] (p-\j-\tt-b) to[out=\myout,in=\myin] (p-\j-\tt-e);
		}}
		\draw[surfacecol,line width=1.2,name path=lower boundary] (a) to[out=\inb0,in=\outb0] (b) to[out=\outb1,in=\inb1] (c);
		\tikzset{
			geodesic style/.style={green!80!black,line width=.6}
		}
		
		\draw[thick, definition, dashed] (1.3,1.7,-.3) arc (180:0:.7 and .3);
		\draw[thick, definition] (1.3,1.7,-.3) arc (180:360:.7 and .3);
		\draw[thick, definition, dashed] (.33,3.27,-.3) arc (190:-10:1.7 and .7);
		\draw[thick, definition] (.33,3.27,-.3) arc (190:350:1.7 and .7);
		%\draw[thick, definition] (2,1.7,-.3) circle (.7 and .3);
%		\draw[thick, definition] (2,3.4,-.3) circle (1.7 and .7);
		\draw[thick, definition, dashed] (2,2.4,-.3) circle (.35 and .15);
		\draw[thick, definition] (1.3,1.7,-.3) -- (1.65, 2.4, -.3);
		\draw[thick, definition] (2.7,1.7,-.3) -- (2.35, 2.4, -.3);
		\draw[thick, definition] (1.3,1.7,-.3) -- (.37, 3.2, -.3) node[left] {\footnotesize\(\partial J^+(W)\)};
		\draw[thick, definition] (2.7,1.7,-.3) node[right] {\footnotesize\(W\)} -- (3.63, 3.2, -.3);
		
		\node[left] at (a) {\footnotesize\(S\)};
		
		
		\end{tikzpicture}
		\column{0.5\textwidth}
		\centering
		\(\frac{\dot{A}}{A}<-w\implies \lambda\ge \frac{D-2}{w}\) past a focal point \(p\).\\
		\(\overrightwitchonpitchfork*{\text{\tiny Reductio ad Absurdum}}\) \\ %\(\mathwitch*\) \\ %Reductio ad Absurdum
		\(\ell \cap \partial J^+(W)\) is compact.\\
		\(\pumpkin\Downarrow\pumpkin\) \\ 
		\(\partial J^+(W)\) is compact.\\
		\(\mathghost\Downarrow\mathghost\) \\ 
		Absurd as the Cauchy hypersurface S is non-compact.
	\end{columns}
	\begin{columns}
		\column{0.5\textwidth}
		\begin{ideablock}{\textcolor{black}{\(\pumpkin\) Fact \( 1 \) \(\pumpkin\)}}
			Any point \(p\in\partial J^+(W)\) is connected to \(W\) by a future-going orthogonal null geodesic.
		\end{ideablock}
	\column{0.5\textwidth}
	\begin{theoblock}{\(\mathghost\) Fact \( 2\) \( \mathghost\)}
		\(\varphi :M\hookrightarrow N\) embedding, \(M\) compact \& \(N\) connected with the same dimension implies that \(\varphi\) is a diffeomorphism, and in particular \(N\) is compact.
	\end{theoblock}
	\end{columns}
\end{frame}

%%%%%%%%%%%%%%%%%%%%%%%%%%%%%%%%%%%%%%%%%%%%%%%%%%%%%%Slide 11

\begin{frame}
	\frametitle{De Sitter Space: a useful example}
	\begin{columns}
		\column{0.5\textwidth}
		\begin{figure}
			\begin{tikzpicture}[scale=0.9]
			\coordinate (a) at (0,0);
			\coordinate (b) at (5,0);
			\coordinate (c) at (5,5);
			\coordinate (d) at (0,5);
			\coordinate (q) at (3,1);
			\draw[thick, definition] (a) -- (b) -- (c) -- (d) -- (a);
			\draw[thick,definition] (a) -- (c);
			\draw[definition] (b) -- (d);
			\fill[definition, opacity = 0.22] (a) -- (b) -- (c) -- (a);
			\draw[definition] (0, 2.5) -- (5, 2.5);
			\draw[thick, definition] (a) to [out = 18, in = 180, looseness = 0.5] (5,2);
			\draw[thick, definition, dashed] (a) to [out = 9, in = 180, looseness = 0.5] (5,1);
			\draw[thick, definition, dashed] (a) to [out = 27, in = 180, looseness = 0.5] (5,3);
			\draw[thick, definition, dashed] (a) to [out = 36, in = 180, looseness = 0.5] (5,4);
			\draw[thick, titleorange, dashed, postaction={decorate,decoration={markings,mark=at position .75 with {\arrow{stealth}}}}] (4,0) -- (0,4) node[midway, above] {\(\gamma\)};
			
			
			\fill[definition] (q) circle(1.1pt);
			\node[below] at (q) {\(q\)};
			\node[left] at (0, 2.5) {\(\hat{M}\)};
			\node[right] at (5, 2.5) {\(M\)};
			\end{tikzpicture}
			\caption{Penrose diagram of de Sitter space.}
		\end{figure}
		\column{0.5\textwidth}
		\begin{defblock}{Metric of complete de Sitter space}
			\(ds^2 = -dt^2 + R^2\cosh^2\left(\frac{t}{R}\right)d\Omega^2\)
		\end{defblock}
		\begin{defblock}{Metric of partial de Sitter space}
			\(ds^2 = -dt^2 + R^2exp\left(-\frac{2t}{R}\right)d\vec{x}^2\)
		\end{defblock}
	\end{columns}
\end{frame}

%%%%%%%%%%%%%%%%%%%%%%%%%%%%%%%%%%%%%%%%%%%%%%%%%%%%%%Slide 12

\section{Black Holes: a physical application}
\subsection{Cosmic Censorship}

\begin{frame}
		\frametitle{Weak Cosmic Censorship}
		\begin{defblock}{Cosmic Censorship}
			In any \emph{localized} process, in an asymptotically Minkowskian spacetime, the region in the far distance and to the far future continues to exist, just as in Minkowski space.
			
			The evolution seen by an outside observer is supposed to be predictable based on the classical Einstein equation.
		\end{defblock}
	\vfill
	\pause
	\begin{columns}
			\column{0.5\textwidth}
			\centering
			%\includegraphics[scale=0.8]{example-image-duck}
			%todo: mettici una figura tipo la 36 a pagina 62
			\begin{tikzpicture}[scale=0.5]
				\coordinate (a) at (3.7, -2);
				\coordinate (b) at (5, 3);
				\filldraw[definition] (0,0) ellipse (2 and 1);
				\draw[thick, theorem] (a) to[out = 110, in = - 135] (b);
				\draw[titleorange] (2,0) -- (b) node[midway, above] {\(\gamma\)};
				
				\node[right] at (a) {\footnotesize\(\mathcal{I}\)};
				\node[below] at (0, -1) {\footnotesize\(W\)};
				
			
				
			\end{tikzpicture}
			\column{0.5\textwidth}
			\begin{defblock}{Black hole region}
				\centering
				\large
				\(B \coloneqq M \setminus J^-(\mathcal{I})\)
			\end{defblock}
\end{columns}

\end{frame}

%%%%%%%%%%%%%%%%%%%%%%%%%%%%%%%%%%%%%%%%%%%%%%%%%%%%%%Slide 13
\subsection{Black Holes}
	\begin{frame}
		\begin{theoblock}{Existence of Black Holes}
			Any compact trapped surface \(W\) is in the black hole region \(B\).
			
			As a consequence, in any asymptotically flat spacetime that contains a trapped surface, a black hole forms.
		\end{theoblock}
		
		%todo: qui ricordati che flat spacetime + global. hyper. ==> esiste Cauchy hypersurf. non compatta
		\vfill
		\pause
		\begin{columns}
			\column{0.6\textwidth}
			\begin{itemize}
				\item \(J^+(W\subseteq B) \subseteq B\);
				\item Black holes cannot split.
				\item The horizon generators are prompt null geodesics, everywhere contained in the horizon.
			\end{itemize}
			
			\column{0.4\textwidth}		
			%\includegraphics[scale=0.7]{example-image-duck}
			%todo: qui ci vuole una figura tipo la 38 di pagina 63.
			\begin{tikzpicture}[scale=0.5]
				\coordinate (a) at (0,0);
				\coordinate (b) at (7,0);
				\coordinate (c) at (8,2);
				\coordinate (d) at (1,2);
				
				\coordinate (e) at (0,5);
				\coordinate (f) at (7,5);
				\coordinate (g) at (8,7);
				\coordinate (h) at (1,7);
				
				\draw (a) -- (b) -- (c) -- (d) -- (a);
				\draw (e) -- (f) -- (g) -- (h) -- (e);
				
				\draw[thick, definition, dashed] (2.5,1) arc (180:0:1.5 and .5);
				%\draw[thick, definition] (2.5,1) arc (180:360:1.5 and .5);
				
				%\draw[thick, definition] (2,6) ellipse (.7 and .3);
				%\draw[thick, definition] (6,6) ellipse (.9 and .4);
				\draw[thick, definition] (1.3, 6) arc (180:0:.7 and .3);
				\draw[thick, definition] (5.1, 6) arc (180:0:.9 and .4);
				\filldraw[thick, definition, fill opacity = .22] (2.5, 1) to[out = 90, in = -60] (1.3, 6) arc (180:360:.7 and .3) to[out = -80, in = -110, looseness = 3.0] (5.1, 6) arc (180:360:.9 and .4) to[out = -110, in = 90] (5.5, 1) arc (0:-180:1.5 and .5);
				%\draw[thick, definition] (5.5, 1) to[out = 90, in = -110] (6.9, 6);
				%\draw[thick, definition] (2.7, 6) to[out = -80, in = -110, looseness = 3.0] (5.1, 6);
				
				\node[right] at (b) {\(S\)};
				\node[right] at (f) {\(S'\)};
				\node[left] at (2.5, 1) {\(B\)};

			\end{tikzpicture}
		\end{columns}
	\end{frame}

%%%%%%%%%%%%%%%%%%%%%%%%%%%%%%%%%%%%%%%%%%%%%%%%%%%%%%Slide 14
\subsection{Black Hole Area Theorem}
\frametitle{Black Holes Area Theorem}
\begin{frame}
	\begin{theoblock}{Hawking area theorem}
		The area of the black hole horizon can only increase. 
		
		%The theorem applys separately to each component of the black hole region. 
		If two or more black holes merge, they produce a black hole whose horizon area is at least equal to the sum of the horizon areas of the original black holes.
	\end{theoblock}

	\pause
\begin{columns}
	\column{0.6\textwidth}
	\centering
	\(\overrightflutteringbat{\small\text{Reductio ad Absurdum}}\)\\
	Define \(W'\) by pushing \(W\) out a little in the region where \(\theta < 0\).\\
	\(\Downarrow\)\\
	\(\ell\) prompt connects \(\mathcal{I}\) to a point of \(W'\) outside \(B\).\\
	\(\Downarrow\)\\
	Contradiction: \(\mathcal{I}\) is arbitrarily far away but \(\ell\) is prompt only for a bounded affine distance.
	\column{0.4\textwidth}		
	%\includegraphics[scale=0.7]{example-image-duck}
	%todo: qui serve la figura 40 di pagina 66
	
	\begin{tikzpicture}[scale=0.7]
		\draw[titleorange, thick, dashed] (30:2) to[out = -30, in= 30, looseness = 2] (-30:2);
		
		\fill[idea, opacity = 0.22] (30:2) to[out = -30, in= 30, looseness = 2] (-30:2) arc(-30:30:2);
		\filldraw[thick, theorem, fill opacity = 0.22] (0,0) circle (2);
		\node[above] at (0,2) {\(W\)};
		\node[right] at (2.7,0) {\(W'\)};
		
		
	\end{tikzpicture}
	
\end{columns}

\end{frame}

\begin{frame}
	\centering
	
	{\Huge \textcolor{theorem}{Thank you for your attention!}}
	\vskip 17pt
	\nocite{*}
	\textbf{Bibliography}
	\printbibliography
\end{frame}

\end{document}
